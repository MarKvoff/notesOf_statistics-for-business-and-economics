\documentclass[a4paper]{ctexrep}

% --------- start of config ---------
% 宏包引用
\usepackage{listings}
\usepackage{geometry}
\usepackage{hyperref}
\usepackage{graphicx}
\usepackage{tcolorbox}
\usepackage{amsmath}
\usepackage{amssymb}

% 设置页边距
\geometry{a4paper,left=2cm,right=2cm,top=1.5cm,bottom=1.5cm}
% use url highlight
\hypersetup{
    colorlinks=true,
    linkcolor=blue,
    filecolor=magenta,
    urlcolor=cyan,
}
\urlstyle{same}
% 行间距
\renewcommand{\baselinestretch}{1.2}
% 段间距
\setlength{\parskip}{1em}
% 设置图片文件夹
\graphicspath{{./img/}}
% 封面信息
\title{Notes of <statistics for business and economics>}  % 文章标题
\author{ZHEMIN CHEN}   % 作者的名称
\date{February 9, 2021}   % 创建日期
% --------- end of config ---------

\begin{document}
\maketitle
\noindent
Last edit: \today

This is a concise notes for the book <statistics for business and economics>.

\tableofcontents
\clearpage

\chapter{数值方法}
本章包括位置、离散程度、形态和相关程度的数值度量。

\section{位置的度量}

\subsection{平均数}
平均数(mean)提供了数据中心位置的度量。$\bar{x}$表示样本数据的平均值,$\mu$表示总体数据的
平均值。平均值的数据位置描述容易受到极端值的影响。

\textbf{样本平均数}
\[\bar{x}=\frac{\sum x_{i}}{n}\]

\textbf{总体平均数}
\[\mu =\frac{\sum x_{i}}{N}\]

\subsection{加权平均数}
计算平均数时加入对每个观测值赋予显示其重要性的权重。

\textbf{加权平均数}
\[\bar{x}=\frac{\sum w_{i}x_{i}}{\sum w_{i}}\]
式中,$w_{i}$是第$i$个观测值的权重。

\subsection{中位数}
中位数(median)是对数据中心位置的另一种度量。

将数据按升序排列,

(a)对奇数个观测值,中位数是中间的数值。

(b)对偶数个观测值,中位数是中间两个数值的平均值。

虽然在度量数据的中心位置时更加常用的是平均数,但是平均数往往会受到异常大或小的数值影响。在数据集
含有异常值的情况下,中位数往往更加适合于度量数据的中心位置。

\subsection{几何平均数}
几何平均数(geometric mean)是一种位置的度量。经常用于分析财务数据的增长率。平均数只适合于加法
过程。对于乘法过程,诸如增长率的应用,几何平均数是合适的位置度量。当你需要确定过去几个连续时期的
平均变化率时,都能应用几何平均数。

\textbf{几何平均数}
\[\bar{x}_{g} = \sqrt[n]{x_{1}x_{2}\cdots x_{n}}\]

\subsection{百分位数}
提供了数据如何散布在从最小值与最大值的区间上的信息。对于包含$n$个观测值的数据集,第$p$百分位数将数据分割成为两个部分:
大约有$p\%$的观测值比第$p$百分位数小;大约有$(100-p)\%$的观测值比第$p$百分位数大。

将$n$个观测值从小到大排列,最小的在第1个位置,以此类推。$L_{p}$表示第$p$百分位数的位置,计算公式如下:
\begin{tcolorbox}[title = {第$p$百分位数位置}]
\[L_{p}=\frac{p}{100}(n+1)\]
\end{tcolorbox}

人们经常需要将数据分为四部分,每部分包含大约1/4的观测值。这些分割点称为四分位数:
\begin{tcolorbox}[title = {四分位数}]
$Q_{1}=$第一四分位数,或第25百分位数

$Q_{2}=$第二四分位数,或第50百分位数(也是中位数)

$Q_{3}=$第三四分位数,或第75百分位数
\end{tcolorbox}

\section{变异程度的度量}
除了位置的度量以外,人们往往还需要考虑变异程度亦即离散程度的度量。

\subsection{四分位数间距}
四分位数间距(interquartile range,IQR)作为变异程度的一种度量,能够克服异常值的影响。
它是$Q_{3}$与$Q_{1}$的差值,是中间50\%数据的极差。
\begin{tcolorbox}[title = {四分位数间距}]
\[IQR=Q_{3}-Q_{1}\]
\end{tcolorbox}

\subsection{方差}
方差(variance)是用所有数据对变异程度所做的一种度量。方差依赖于每个观测值$(x_{i})$与平均值
之间的差异。

如果数据来自总体,对于有N个观察值的总体,用$\mu$表示总体平均数,总体方差的定义如下:

\[\sigma^{2}=\frac{\Sigma(x_{i}-\mu)^{2}}{N}\]

大多数的统计应用需要分析样本数据。当计算样本方差时,更希望用它来估计总体方差$\sigma^{2}$.
用$s^{2}$表示的n个样本数据的方差定义如下:

\[s^{2} = \frac{\Sigma(x_{i} - \bar{x})^{2}}{n - 1}\]

方差的平方单位使得人们对于方差的数值很难找到直观的理解和诠释。但是方差在比较两个或两个以上变量
的变异程度时很有用。

\subsection{标准差}
定义标准差(standard deviation)为方差的正平方根:

\[s = \sqrt{s^{2}}\]
\[\sigma = \sqrt{\sigma^{2}}\]

标准差和原始数据的单位相同,因此标准差更容易与平均数和其他与原始数据有相同测量单位的统计数据进行
比较。

\subsection{标准差系数}
当对标准差相对于平均数大小的描述统计量感兴趣时,可以使用标准差系数(coefficient of variation):

\[(\frac{\mbox{标准差}}{\mbox{平均数}}\times 100)\%\]

在比较具有不同标准差和不同平均数的变量的变异程度时,标准差系数是一个很有用的统计量。


\section{分布形态,相对位置的度量}

\subsection{分布形态}
数据的相对频数分布直方图是对分布形态很好的一种图形描述。分布形态的一种重要数值度量被称为偏度(skewness)。
如果数据时对称的,偏度为0,平均数与中位数也相等。当数据严重偏离时,中位数是对数据中心位置的首选度量。

\subsection{z-分数}
一个数据集,除了位置、变异程度和形态之外,数据集中数值的相对位置往往也很有用。相对位置的度量能帮助确定一个
特殊数值的数据距离平均数有多远。

假设在$n$个观测值$x_{1}, x_{2}, \cdots, x_{n}$下,$s$为标准差,$x_{i}$的z-分数定义为:

\[z_{i}=\frac{x_{i}-\bar{x}}{s}\]

z-分数往往被称为标准化数值,被解释为$x_{i}$与平均数$\bar{x}$的距离是$z_{i}$个标准差。一个变量的数值转换成
z-分数的过程常常被称为z变换。

\subsection{切比雪夫定理}
该定理能使我们指出与平均数的距离在某个特定个数的标准差之内的数据值所占的比例:

与平均数的距离在$z$个标准差之内的数据值所占的比例至少为$(1-1/z^{2})$,其中$z$是大于1的任意实数。

\subsection{经验法则}
切比雪夫定理的优点之一是它适用于任何数据集而不论其数据分布的形状。在实际应用中,大部分数据具有对称的峰形或钟形分布。
当数据被认为近似于这种分布时就可以运用经验法则(empirical rule)来确定与平均数的距离在某个特定个数的标准差之内的
数据值所占的比例。
\begin{tcolorbox}[title = {经验法则}]
对于具有钟形分布的数据:
\tcblower
$\bullet$大约68\%的数据值与平均数的距离在1个标准差之内。

$\bullet$大约95\%的数据值与平均数的距离在2个标准差之内。

$\bullet$几乎所有的数据值与平均数的距离在3个标准差之内。
\end{tcolorbox}


\section{两变量间关系的度量}
本节将介绍描述两变量间关系的度量:协方差和相关系数。

\subsection{协方差}
对于一个容量为$n$的样本,其观测值为$(x_{1},y_{1}),(x_{2},y_{2}),\cdots,(x_{n},y_{n})$,样本协方差定义为:
\begin{tcolorbox}[title = {样本协方差}]
\[s_{xy}=\frac{\Sigma(x_{i}-\bar{x})(y_{i}-\bar{y})}{n-1}\]
\end{tcolorbox}
协方差是两变量线性关系的度量。$s_{xy}$为正值且越大表示$x$和$y$之间存在越强的正的线性关系;$s_{xy}$为负值且越小
表示$x$和$y$之间存在越强的负的线性关系;$s_{xy}$接近于0表示$x$和$y$几乎不存在线性关系。

\subsection{相关系数}
在使用协方差作为线性关系强度的度量时,一个问题就在于协方差的值依赖于$x$和$y$的计量单位。同样的一组观测数据
在使用不同的计量单位下协方差所计算出来的值也会不同,但是其线性关系强度理应相同。为了避免这种情况,我们将使用
相关系数(correlation coefficient)对两变量间的相关关系进行度量。
\begin{tcolorbox}[title = {皮尔逊积矩相关系数:样本数据}]
\[r_{xy}=\frac{s_{xy}}{s_{x}s_{y}}\]
式中,$r_{xy}$为样本相关系数;$s_{xy}$为样本协方差;$s_{x}$为$x$的样本标准差;$s_{y}$为$y$的样本标准差
\end{tcolorbox}
相关系数的范围是$-1\sim+1$。当相关系数接近于$-1$或者$+1$时,表示较强的线性关系,而越接近于0,线性关系也越弱。

相关系数提供了线性但不是因果关系的一个度量。两个变量之间较高的相关系数,并不意味着一个变量的变化会引起另一个变量的变化。


\chapter{概率}
概率是对事件发生的可能性的数值度量。

\section{随机试验、计数法则和概率分配}
研究概率时,处理的试验具有以下特征:

1. 试验结果是确定的,在许多情形下试验结果甚至在进行试验前就已经列出。

2. 在任意一次试验或者重复中,有且仅有一种可能的试验结果发生。

3. 试验中究竟哪种试验结果出现完全由偶然性决定。

这样类型的试验被称为随机试验(random experiment)。
\begin{tcolorbox}[title = {随机试验}]
随机试验是一个过程,它所产生的试验结果是完全确定的。在每一次重复或者试验中,出现哪种结果完全由偶然性来决定。
\end{tcolorbox}

一旦确定了试验的所有可能结果,就确定了随机试验的样本空间(sample space)。
\begin{tcolorbox}[title = {样本空间}]
随机试验的样本空间是试验所有结果组成的一个集合。
\end{tcolorbox}
一种特定的试验结果被称为样本点(sample point),它是样本空间的一个元素。例如掷一个六面骰子,试验结果定为
每次抛掷停置后面朝上的点数,那么该随机试验的样本空间有6个样本点,为:
\[S=\{1,2,3,4,5,6\}\]

\subsection{计数法则、组合和排列}
多步骤试验计数法则适用于多步骤试验。例如抛掷两枚硬币,试验结果是两枚硬币朝上一面的图案,该试验可以被视为
一个两步骤的试验:第一步抛掷第一枚硬币,第二步抛掷第二枚硬币。记$H$为正面朝上,$T$为反面朝上,则该试验
的样本空间为:
\[S=\{(H,H),(H,T),(T,H),(T,T)\}\]
\begin{tcolorbox}[title = {多步骤试验计数法则}]
如果一个试验可以看作循序的$k$个步骤,第一步有$n_{1}$种试验结果,第二步有$n_{2}$种试验结果,以此类推。
那么试验结果的总数为$n_{1}\times n_{2}\times \cdots \times n_{k}$。
\end{tcolorbox}

组合:在从$N$项中选取$n$项的试验中,组合计数法则可以用于确定试验结果的数目。
\begin{tcolorbox}[title = {组合计数法则}]
从$N$项中任取$n$项的组合数为:
\[C^{N}_{n}=\binom{N}{n}=\frac{N!}{n!(N-n)!}\]
\end{tcolorbox}

排列:当从$N$项中选取$n$项并且考虑选取的顺序时(对于选出的$n$项,不同的选取顺序被认为是不同的试验结果),
排列可以用于计算有多少种不同的试验结果。
\begin{tcolorbox}[title = {排列计数法则}]
从$N$项中任取$n$项的排列数为:
\[P^{N}_{n}=n!\binom{N}{n}=\frac{N!}{(N-n)!}\]
\end{tcolorbox}

\subsection{概率分配}
古典法、相对频率法和和主观法是最为常用的概率分配法则。所有概率分配方法都必须满足概率分配的基本条件。
\begin{tcolorbox}[title = {概率分配的基本条件}]
1. 分配给每个试验结果的概率值都必须在0和1之间。以$E_{i}$表示第$i$种试验结果,$P(E_{i})$表示这种
结果发生的概率,则:
\[0 \leqslant P(E_{i}) \leqslant 1\]

2. 所有的试验结果的概率之和必须为1.对$n$个试验结果的情形有:
\[P(E_{1}) + P(E_{2}) + \cdots + P(E_{n}) = 1\]
\end{tcolorbox}


\section{事件及其概率}
给出与样本点相对应的事件(event)一个正式的定义:
\begin{tcolorbox}[title = {事件}]
事件是样本点的一个集合。
\end{tcolorbox}

\begin{tcolorbox}[title = {事件的概率}]
事件的概率等于事件中所有样本点的概率之和。
\end{tcolorbox}

只要能够确认一个试验的所有样本点并且为其分配概率,就能够根据定义来计算某一事件的概率。

\section{概率的基本性质}
\subsection{事件的补}
给定一个事件$A$, 定义事件$A$的补为“所有不包含在事件$A$中的样本点”,记为$A^{c}$。
\begin{tcolorbox}[title = {利用事件的补计算概率}]
\[P(A)=1-P(A^{c})\]
\end{tcolorbox}

\subsection{加法公式}
\begin{tcolorbox}[title = {两个事件的并}]
$A$和$B$的并是所有属于$A$或$B$,抑或同时属于二者的样本点构成的事件,记作$A\cup B$。
\end{tcolorbox}

\begin{tcolorbox}[title = {两个事件的交}]
给定两个事件$A$和$B$,则$A$和$B$的交是同时属于$A$和$B$的样本点构成的事件,记作$A\cap B$。
\end{tcolorbox}

\begin{tcolorbox}[title = {加法公式}]
\[P(A\cup B)=P(A)+P(B)-P(A\cap B)\]
\end{tcolorbox}

\begin{tcolorbox}[title = {互斥事件}]
如果两个事件没有公共的样本点,则称这两个事件互斥,即$P(A\cap B) = 0$
\end{tcolorbox}

\section{条件概率}
某个事件发生的可能性经常会受到另一个相关事件发生与否的影响。假设事件$A$发生的概率为$P(A)$,如果
获得了新的信息确知另一个相关事件$B$已经发生了,我们希望利用这一新的信息来重新计算事件$A$发生的可能性。
则事件$A$发生的可能性叫做条件概率,记作$P(A|B)$。
\begin{tcolorbox}[title = {条件概率}]
\[P(A|B)=\frac{P(A\cap B)}{P(B)}\]
\end{tcolorbox}

\subsection{独立事件}
如果事件$A$的概率不会由于事件$B$的存在而改变,则称事件$A$和$B$是独立事件。
\begin{tcolorbox}[title = {独立事件}]
两个事件$A$和$B$是互相独立的,如果
\[P(A|B) = P(A)\]
或
\[P(B|A)=P(B)\]
否则,两个事件是相依的。
\end{tcolorbox}

\subsection{乘法公式}
加法公式是用来计算两个事件的并的概率,而乘法公式则是用来计算两个事件的交的概率。
\begin{tcolorbox}[title = {乘法公式}]
\[P(A\cap B) = P(B)P(A|B)\]
或
\[P(A\cap B) = P(A)P(B|A)\]
\end{tcolorbox}
乘法公式与条件概率看似是一个公式,但是视角不同。乘法公式的意义在于两个事件同时发生的概率等于
其中一个事件先单独发生的概率乘上剩下的那个事件在前一个事件已经发生的情况下发生的概率。

当两个事件是独立事件时,乘法公式可以简化成一个特殊的形式。
\begin{tcolorbox}[title = {独立事件的乘法公式}]
\[P(A\cap B) = P(A)P(B)\]
\end{tcolorbox}
不要混淆互斥事件和独立事件,这是两个不同的概念。两个概率不为0的事件不可能即是互斥事件,又是独立事件。
如果两个互斥的事件之一被确认已经发生,那么另一个事件不会发生,从而另一个事件发生的概率降为0,因此
这两个事件是相依的。

\section{贝叶斯定理}
todo


\chapter{离散型概率分布}


\section{随机变量}
随机变量提供了用数值描述试验结果的方法,随机变量的取值必须是数值。
\begin{tcolorbox}[title = {随机变量}]
随机变量是对试验结果的数值描述。
\end{tcolorbox}
随机变量将每一个可能出现的试验结果赋予一个数值,随机变量的值取决于试验结果。可以取有限多个值或者
无限可数个值的随机变量称为离散型随机变量。可以取某一区间或多个区间内任意值的随机变量称为连续型
随机变量是对试验结果的数值描述。

\section{离散型概率分布}
随机变量的概率分布是描述随机变量取不同值的概率。对于离散型随机变量$x$,概率函数给出随机变量取每种
值的概率,记作$f(x)$。一个离散型随机变量的概率函数必须满足如下两个条件:
\begin{tcolorbox}[title = {离散型概率函数的基本条件}]
\[f(x)\geqslant 0\]
\[\Sigma f(x) = 1\]
\end{tcolorbox}
常用的离散型随机变量的概率分布通常以公式的形式给出,二项分布、泊松分布和超几何分布是其中最重要的三类分布。

















\end{document}